\section{Implementation}

Projektet består i grunden av två huvudsakliga implementationer, vilka är databasimplementationen samt GUI implementationen. Dessa implementationer har olika uppgifter i projektet. Databasen kommer stå för sparandet av viktig information samt att kunna få ut det man vi ha ut ur informationen. GUIt ska göra programmet användarvänligt så att även den minst tekniske ska kunna använda det utan att ha någon förkunskap om funktionaliteten.

\subsection{Databasimplementation}

Databasen vi implementerade är av typen MySQL och går att hitta på institutionens egna server som de tillhandager EDA216 studenter. Serverns adress är puccini.cs.lth.se. För att göra våra querys säkra mot attacker använder vi oss av preparedStatements, vilka vi skriver i php. 

Eftersom att flera ska vara inne samtidigt och utföra handlingar i databasen inte var ett krav så beslöt vi oss för att inte implementera transactions. All SQL kod vi använt för att skapa databasen går att hitta under \emph{6. Databasdump}.

\subsection{GUI Implementation}



Resultatet av hela projektet. Vad vi klarat av att implementera och vad vi inte gjort. Detta för att se hur mycket vi gjort. 